\documentclass[letter, 11pt]{article}
% Change "article" to "report" to get rid of page number on title page
\usepackage{amsmath,amsfonts,amsthm,amssymb}
\usepackage{setspace}
\usepackage{graphicx,float,wrapfig}
%\usepackage{parskip}
\usepackage{enumerate}

\usepackage{fourier}
\usepackage[T1]{fontenc}
\usepackage[protrusion=true,expansion=true]{microtype}

% In case you need to adjust margins:
\topmargin=-0.45in      %
\evensidemargin=0in     %
\oddsidemargin=0in      %
\textwidth=6.5in        %
\textheight=9.0in       %
\headsep=0.25in         %
\parindent=0in

\usepackage[nodayofweek]{datetime} \usdate
% Pdf metadata
\pdfinfo{  /Author (Blake Riley)
           /Title (Econ 533 Handout 1)
           /Keywords ()
           /ModDate (D:\pdfdate)}

\newtheoremstyle{basic}% name
   {5pt}% Space above
   {5pt}% Space below
   {\itshape \leftskip=1em}% Body font
   {-1em}% Indent amount
   {\bfseries}% Theorem head font
   {:}% Punctuation after theorem head
   { }% Space after theorem head
   {}% Theorem head spec (can be left empty, meaning `normal')
\theoremstyle{basic}         
\newtheorem{exercise}{Exercise}[section]
\newtheorem{definition}{Definition}[section]
\newtheorem{theorem}{Theorem}[section]
\newtheorem{lemma}[theorem]{Lemma}


%%%%%%%%%%%%%%%%%%%%%%%%%%%%%%%%%%%%%%%%%%%%%%%%%%%%%%%%%%%%

% Custom commands

\newcommand{\R}{\mathbb{R}}
\newcommand{\N}{\mathbb{N}}
\newcommand{\E}{\operatorname{E}}
\renewcommand{\P}{\operatorname{Pr}}
\newcommand{\Var}{\operatorname{Var}}
\newcommand{\Cov}{\operatorname{Cov}}
\newcommand{\cond}{\,|\,}
\newcommand{\bigcond}{\;\big|\;}
\newcommand{\argmax}{\mathop{\operatorname{arg\,max}}}
\newcommand{\noti}{{{\scriptscriptstyle-}\!i}}
\newcommand{\notj}{{{\scriptscriptstyle-}\!j}}
\newcommand{\notij}{{{\scriptscriptstyle-}\!\{i,j\}}}
\newcommand{\I}{\mathbb{I}}

%%%%%%%%%%%%%%%%%%%%%%%%%%%%%%%%%%%%%%%%%%%%%%%%%%%%%%%%%%%%%
%%%%%%%%%% The main document content
%%%%%%%%%%%%%%%%%%%%%%%%%%%%%%%%%%%%%%%%%%%%%%%%%%%%%%%%%%%%%

\begin{document}
\begin{spacing}{1.0}

\noindent
\textbf{Handout 1} \\
Econ 533 \\
January 30, 2015 \\
TA: Blake Riley \\

\begin{center}
{\Large Microeconomic Theory II: Spring 2015}
\end{center}

\section{Static games of complete information}

\subsection{Elements of a static game of complete
  information}

\begin{definition}[Normal form game]
  A game with $N$ players in \textbf{normal form
    representation} $\Gamma_N$ is a collection of players
    $N$, strategy sets $S_i$ and payoff functions
    $u_i(\cdot)$ where $u_i \,:\, \prod_{j=1}^N S_j \to
    \R$, i.e. $\Gamma_N = \left(N, \{S_i, u_i(\cdot)\}_{i=1}^N\right)$.
\end{definition}

\begin{definition}[Complete information]
  Each player's payoff function is commonly known among everyone.
\end{definition}

The elements $s_i \in S_i$ are called agents $i$'s pure strategies. In a
normal form game, the players choose their strategies simultaneously. This
does not necessarily imply that the parties act simultaneously.

\subsection{Strategies}

Strategies are complete plans of action, containing all possible decisions
relevant to a player's actions. Imagine strategies as a book of
instructions detailed enough for someone else to play as your proxy without
needing additional information. This will be more important to remember
when we consider dynamic and Bayesian games.

\begin{definition}[Domination]
  Given two strategies $s_i, s'_i \in S_i$ in the
  normal form game $\Gamma_N$, we say $s_i$ \textbf{strictly dominates}
  $s'_i$ if $\forall s_\noti \in S_\noti = \prod_{j\neq i} S_j$, player $i$ has
  $u_i(s_i,s_\noti) > u_i(s'_i, s_\noti)$. The strategy $s_i$ \textbf{weakly
    dominates} $s'_i$ if $\forall s_\noti \in S_\noti$,
  player $i$ has $u_i(s_i,s_\noti) \ge u_i(s'_i, s_\noti)$ and $\exists
  s^*_\noti \in S_\noti$ such that $u_i(s_i, s^*_\noti) > u_u(s'_i,
  s^*_\noti)$. The strategy $s_i$ \textbf{very weakly
    dominates} $s'_i$ if $\forall s_\noti \in S_\noti$,
  player $i$ has $u_i(s_i,s_\noti) \ge u_i(s'_i, s_\noti)$.
\end{definition}

\begin{definition}[Dominant strategies]
  A strategy $s_i \in S_i$ is \textbf{strictly (weakly, very weakly)
    dominant} if $s_i$ strictly (weakly, very weakly) dominates all other
  strategies in $S_i$.
\end{definition}

Notice each player can only have a single strictly or weakly dominant
strategy by these definitions, but might have many very weakly dominant
strategies. Some other sources will refer to very weak dominance as weak dominance.

\begin{definition}[Mixed strategy]
  Given player $i$'s finite pure
  strategy set $S_i$, a \textbf{mixed strategy} for player $i$ is
  a function $\sigma_i : S_i \to [0,1]$ which assigns
  each pure strategy a probability that it will be
  played, where $\sum_{s_i\in S_i} \sigma_i(s_i) = 1$.
\end{definition}

Once we use mixed strategies, the basic notion of a game in normal form has
to be adjusted to $\Gamma_N = \left(N, \{\Delta(S_i),
  u_i(\cdot)\}_{i=1}^N\right)$, where $\Delta(S)$ is the probability
simplex over set $S$. Pure strategies can now be considered as degenerate
mixed strategies. The notions of dominance carry over to mixed
strategies.

\hspace{1em}
By iteratively removing dominated strategies, we end up with a refined set
of strategies that can plausibly be chosen by each player. This refined set
is our first example of a solution concept.

\begin{theorem}[Facts about iterated removal of strictly dominated
  strategies]
  For any game with compact strategy spaces and continuous payoff functions:
  \begin{enumerate}[a)]\leftskip = 1em
  \item The set of surviving strategies is compact and
    nonempty (so it's meaningful to talk about).
  \item The order of deletion does not matter.
  \item Common knowledge of rationality in a game of complete infomation is
    sufficient to ensure players will only use surviving strategies and
    have common knowledge of that fact.
  \end{enumerate}
\end{theorem}

However, if weakly dominated strategies are also eliminated, then
order might matter and it sometimes cannot be common knowledge that players
will not use weakly dominated strategies.

\begin{definition}
  A game is \textbf{dominance solvable} if iterated deletion of strictly
  dominated strategies leaves a single strategy profile. Sometimes,
  dominance solvable is also used to refer to a single strategy profile
  left after iterated deletion of all weakly dominated strategies at each
  stage.
\end{definition}

\begin{lemma}
  If a pure strategy $s_i$ is strictly dominated, then so is any mixed
  strategy that plays $s_i$ with positive probability.
\end{lemma}

\begin{definition}[Best response]
  A strategy $\sigma_i$ is a \textbf{best response} for
  player $i$ to his rivals' strategies $\sigma_{-i}$ if $\forall \sigma'_i
  \in \Delta(S_i),\; u_i(\sigma_i, \sigma_{-i}) \geq u_i(\sigma'_i,
  \sigma_{-i})$. Strategy $\sigma_i$ is \textbf{never a best response} if
  there is no $\sigma_{-i}$ for which $\sigma_i$ is a best response.
\end{definition}

\begin{definition}[Rationalizable strategies]
  The strategies in $\Delta(S_i)$ that survive
  the iterated removal of strategies that are never best responses are
  known as \textbf{rationalizable strategies}.
\end{definition}

\begin{theorem}[Facts about rationalizable strategies]
  For any game with compact
  strategy spaces and continuous payoff functions:
  \begin{enumerate}[a)]\leftskip = 1em
  \item The set of rationalizable strategies is compact and
    nonempty (so it's meaningful to talk about).
  \item The order of deletion does not matter.
  \item Common knowledge of rationality and the game structure is
    sufficient to ensure players will only play rationalizable strategies
    and have common knowledge of that fact.
  \end{enumerate}
\end{theorem}

\begin{theorem}[Equivalence of iterated strictly undominated strategies and
  rationalizable strategies]
  If $N=2$, the set of iterated strictly
  undominated strategies is exactly the set of rationalizable
  strategies. Some iteratively strictly undominated strategies may not be
  rationalizable if $N>2$. Some rationalizable strategies may not survive
  iterated deletion of weakly dominated strategies.
\end{theorem}

The non-equivalence of the two concepts when $N>2$ comes from an implicit
assumption that other players randomize independently when using mixed
strategies. This is makes sense if we think the players are
actually using mixed strategies. In other contexts, it might be more
natural to think of players ``really'' using pure strategies, with best-responses to
mixed strategies representing best-responses to probabilistic conjectures
about which pure strategy will be played. Under this interpretation, a player could
plausibly think the strategy choices of other players are correlated. This
leads to the notion of correlated rationalizability. Correlated
rationalizability turns out to be equivalent to iterated deletion of
strictly dominated strategies in general.

\begin{theorem}
  In the first- and second-price bid auctions, all bids are rationalizable.
\end{theorem}

\begin{definition}[Nash equilibrium]
  A mixed strategy profile $\sigma = \left(\sigma_1,
    \ldots, \sigma_N\right)$ is a \textbf{Nash equilibrium} of a game
  if \[\forall i \in N,\; \forall \sigma'_i\in \Delta(S_i),\;\;
  u_i(\sigma_i, \sigma_{-i}) \geq u_i(\sigma'_i, \sigma_{-i}),\] i.e.~if
  every player's strategy is a best response to the other strategies in the
  profile.
\end{definition}

\begin{lemma}
  In Nash equilibrium, players are indifferent between all pure strategies
  they play with positive probability. Also, each pure strategy played with
  positive probability is rationalizable and weakly dominates those never
  played. However, equilibrium strategies might be weakly dominated.
\end{lemma}

\begin{theorem}
  Given a game $\Gamma = \left\{N, \{X_i, u_i\}_{i=1}^N\right\}$ in normal
  form such that
  \begin{enumerate}[a)]\leftskip = 1em
  \item $X_i$ is a nonempty, compact, and convex subset of a finite
    dimensional Euclidean space for every $i$ and
  \item $u_i : \prod_{j=1}^N X_j \to \R$ is quasiconcave and continuous,
  \end{enumerate}
  $\Gamma$ has a Nash equilibrium in pure strategies, i.e. there exists
  $x^* \in \prod_{i=1}^N X_i$ such that $\forall i \in N$ and $\forall
  x_i\in X_i$, $ u_i(x_i^*, x_{-i}^*) \geq u_i(x_i, x_{-i}^*)$.
\end{theorem}

\begin{theorem}[Glicksburg's theorem]
  Given a game where the strategy spaces are
  non-empty, compact subsets of a complete metric space and payoff
  functions are continuous, there exists a mixed-strategy Nash equilibrium.
\end{theorem}

\section{Static games of incomplete information}

In games of incomplete information, the agents do not know the other
agents' preferences or beliefs with certainty. All private information of a
player, including their preferences, private knowledge, beliefs about the
other players' preferences, beliefs about the other players' beliefs, and
so on, is summarized by a variable $\theta_i \in \Theta_i$ known as the
player's type. Static games of incomplete information are commonly called
Bayesian games.

\begin{definition}
  A \textbf{Bayesian game} is a collection $\Gamma_N = \left(N, \{\Delta(S_i),
    u_i(\cdot), \Theta_i\}_{i=1}^N, F(\cdot)\right)$. The type space $\Theta_i$
  is the support of a random variable $\theta_i$ representing the private
  information of agent $i$.  The utility of each agent is $u_i: \left(
    \prod_{j=1}^N S_j\right) \times \Theta_i  \to \R$. Finally, $F(\theta_1, \ldots, \theta_N)$ is the
  joint probability distribution of the $\theta_i$'s and is assumed to be
  common knowledge.
\end{definition}

As long as the type spaces are sufficiently descriptive (including possibly
infinite hierarchies of beliefs), the common knowledge assumption is
without loss of generality. In this context, a strategy for agent $i$ is a
function $s_i(\theta_i)$ that assigns a plan of action for each realization
of the random variable $\theta_i$. This is a common point of confusion!
Denote the set of all such functions as $Z_i$.

\begin{definition}
  A profile of strategies $(s_1(\cdot), \ldots, s_N(\cdot))$ is a
  \textbf{Bayesian Nash Equilibrium} of a Bayesian game if $\forall i \in N$, $\forall s'_i \in Z_i$, and $\forall \theta_i
  \in \Theta_i$ occurring with positive
  probability \[\E_{\theta_{-i}}\left[u_i(s_i(\theta_i), s_{-i}(\theta_{-i}),
    \theta_i) \;|\; \theta_i \right] \geq \E_{\theta_{-i}}\left[u_i(s'_i(\theta_i), s_{-i}(\theta_{-i}),
    \theta_i) \;|\; \theta_i \right], \] where the expectation is taken over
  realizations of the other players' types conditional on player $i$'s type.
\end{definition}

\section{Refinements}

\begin{definition}
  A Nash equilibrium $\sigma$ of a normal form game is
  \textbf{trembling-hand perfect} iff there is some sequence of totally
  mixed strategies $\{\sigma^k\}_{k=1}^\infty$ such that $\forall i,k$,
  $\lim \sigma^k = \sigma$ and $\sigma_i$ is a best response to every
  element of $\sigma_{-i}^k$.
\end{definition}

\begin{lemma}
  If $\sigma$ is a normal form trembling-hand perfect Nash equilibrium, then
  $\sigma_i$ is not a weakly dominated strategy for any $i$. Hence, no weakly
  dominated pure strategy is played with positive probability.
\end{lemma}

\begin{theorem}
  Every game with finite strategy sets in normal form has a trembling-hand
  perfect Nash equilibrium.
\end{theorem}

\section{Misc}

\subsection{Order statistics}

\begin{definition}
  The pdf of $k$-th order statistic of $n$ random variables $X_1, \ldots, X_n$
  with common pdf $f(x)$ and distribution function $F(x)$ is \[f_{X_{(k)}}(x) =
  \frac{n!}{(k-1)!(n-k)!}F(x)^{k-1}\left(1-F(x)\right)^{n-k} f(x)\] In
  particular, the pdf of $n$-th order statistic (i.e. the maximum)
  is \[f_{X_{(n)}}(x) = n F(x)^{n-1} f(x)\] and the pdf of the $n-1$-th order
  statistic is \[f_{X_{(n-1)}}(x) = n(n-1) F(x)^{n-2} \left(1-F(x)\right) f(x)\]
\end{definition}

\subsection{Leibniz integral rule}

The Leibniz rule for differentiating under the integral sign can be useful while
deriving FOCs in problems with uncertainty:

\[\frac{d}{d\theta}\int_{a(\theta)}^{b(\theta)} f(x,\theta)\,dx = \frac{d b(\theta)}{d \theta}\,f(b(\theta),\theta)-\frac{d a(\theta)}{d \theta}\,f(a(\theta),\theta)+ \int_{a(\theta)}^{b(\theta)}\frac{\partial}{\partial \theta}\,f(x,\theta)\,dx\]

%%%%%%%%%%%%%%%
\end{spacing}
\end{document}
%%%%% FIN %%%%%
%%%%%%%%%%%%%%%