\documentclass[letter, 10pt]{article}
% Change "article" to "report" to get rid of page number on title page
\usepackage{mathtools,amsfonts,amsthm,amssymb}
\usepackage{setspace}
\usepackage{graphicx,float,wrapfig}
%\usepackage{parskip}
\usepackage{enumerate}

% In case you need to adjust margins:
\topmargin=-0.45in      %
\evensidemargin=0in     %
\oddsidemargin=0in      %
\textwidth=6.5in        %
\textheight=9.0in       %
\headsep=0.25in         %
%\parindent=0in


%%%%%%%%%%%%%%%%%%%%%%%%%%%%%%%%%%%%%%%%%%%%%%%%%%%%%%%%%%%%%
% Format related commands


% Includes a figure
% The first parameter is the label, which is also the name of the figure
%   with or without the extension (e.g., .eps, .fig, .png, .gif, etc.)
%   IF NO EXTENSION IS GIVEN, LaTeX will look for the most appropriate one.
%   This means that if a DVI (or PS) is being produced, it will look for
%   an eps. If a PDF is being produced, it will look for nearly anything
%   else (gif, jpg, png, et cetera). Because of this, when I generate figures
%   I typically generate an eps and a png to allow me the most flexibility
%   when rendering my document.
% The second parameter is the width of the figure normalized to column width
%   (e.g. 0.5 for half a column, 0.75 for 75% of the column)
% The third parameter is the caption.
\newcommand{\scalefig}[3]{
  \begin{figure}[ht!]
    % Requires \usepackage{graphicx}
    \centering
    \includegraphics[width=#2\columnwidth]{#1}
    %%% I think \captionwidth (see above) can go away as long as
    %%% \centering is above
    %\captionwidth{#2\columnwidth}%
    \caption{#3}
    \label{#1}
  \end{figure}}


\newtheorem{theorem}{Theorem}[section]
\newtheorem{definition}[theorem]{Definition}
\newtheorem{lemma}[theorem]{Lemma}
\theoremstyle{definition}
\newtheorem{exercise}{Exercise}[section]

%%%%%%%%%%%%%%%%%%%%%%%%%%%%%%%%%%%%%%%%%%%%%%%%%%%%%%%%%%%%

% Custom commands

\newcommand{\twopartdef}[4]
{
	\left\{
		\begin{array}{ll}
			#1 & \mbox{if } #2 \\
			#3 & \mbox{if } #4
		\end{array}
	\right.
}

\newcommand{\R}{\mathbb{R}}
\DeclareMathOperator{\E}{E}
\newcommand{\tto}{\twoheadrightarrow}

\DeclareMathOperator{\SPA}{SPA}
\DeclareMathOperator{\FPA}{FPA}

%%%%%%%%%%%%%%%%%%%%%%%%%%%%%%%%%%%%%%%%%%%%%%%%%%%%%%%%%%%%%
%%%%%%%%%% The main document content
%%%%%%%%%%%%%%%%%%%%%%%%%%%%%%%%%%%%%%%%%%%%%%%%%%%%%%%%%%%%%

\begin{document}
\begin{spacing}{1.1}

\noindent
\textbf{Handout 6} \\
Econ 502 \\
March 26, 2012 \\
TA: Blake Riley \\

\begin{center}
{\Large Microeconomic Theory I: Spring 2012}
\end{center}

\section{Auction Theory}

Remember that if $v\in \R^n\,iid \sim F$, then
$P(v_i>v_j, \forall j \ne i) = F(v_i)^{n-1}$, so
the density of the largest order statistic of $n-1$ other
agents is $(n-1)F(v_{(1)})^{n-2} f(v_{(1)})$.


\subsection{First-price Auction}

Denote the second-price auction with reserve price $r$
and entry fee $c$ as $\SPA(r,c)$. We have the following
result:

\begin{theorem}[$\SPA(r,0)$ in dominant strategies] The
  following consitutes a weakly dominant strategy for
  player $i$ in the $\SPA(r,0)$:
  \begin{equation}
    b(v) =
    \begin{cases}
       \mathrm{No} & \mbox{if } v < r \\
      v & \mbox{if } v \ge r
    \end{cases}
  \end{equation}
  where $ \mathrm{No}$ indicates no participation.  
\end{theorem}
section{Second-price auction}

Let $v_0(r,c)$ be the $marginal value$, the valuation for
which a potential bidder is indifferent between
participating and not. This is part of the equilibrium,
and clearly depends on the distribution of values $F$.

\begin{theorem}[$\SPA(r,c)$]
  Assume $0 \le c \le 1-r \le 1$. Then, a BNE of
  the $\SPA(r,c)$ with $n$ bidders consists of each player using the
  following strategy:
    \begin{equation}
    b(v) =
    \begin{cases}
      \mathrm{No} & \mbox{if } v < v_0(r,c) \\
      v & \mbox{if } v \ge v_0(r,c)
    \end{cases}
  \end{equation}
  where $v_0(r,c)$ solves $(v_0 - r)F(v_0)^{n-1} = c$ 
\end{theorem}

\section{First-price Auction}

Denote the first-price auction with reserve price $r$ and
entry fee $c$ as $\FPA(r,c)$. 

\begin{theorem}[Symmetric Equilibrium in $\FPA(0,0)$]
  Consider a $\FPA(0,0)$ with $n$ bidders and iid
  private values with distribution $F$. THen the
  following is a symmetric $BNE$ strategy:
  \begin{equation}
    \label{eq:1}
    b(v) = \int_0^v t \frac{g(t)}{G(v)}\,dt = v -
    \int_0^v \frac{G(t)}{G(v)}\, dt = \E[v_{(1)}\,|\,v_{(1)}<v]
  \end{equation}
  where $G(v) = F(v)^{n-1}$ and $v_{(1)}$ is the first
  order statistic among over $n-1$ agents. Moreover, $b$
  is strictly increasing on $[0,1]$ and is
  differentiable.
\end{theorem}

\section{Optimal Auctions}

\begin{theorem}[Revenue Equivalence]
  Suppose bidders are risk neutral and have iid private
  values. Then, any two equilibria of any two auctions that generate
  the same allocation rule and the same conditional
  expected payoffs for each buyer with value $0$ produce
  the same expected revenue for the seller.
\end{theorem}

Myerson (1981) analyzes the first price auction and asks
which mechanism maximizes the revenue of the seller among
all feasible mechanism (where feasible means
implementable in Bayesian Nash equilibrium). As we have
discussed, we can invoke the revelation principle  to
focus on the set of direct revelation mechanisms that are
incentive compatible and individually rational. A direct
revelation mechanism in this context is a pair of outcome
functions $(p,x)$ such that $p:\Theta\to \R^n$ produces a
vector where $p_i$ is the probability agent $i$ gets the
item and $x:\Theta\to\R^n$ represents the bids of the
agents. The expected utility of a bidder with
$\theta_i\in[a_i,b_i]$ is
\begin{equation}
  \label{eq:2}
  U_i(p,x,\theta_i) = \int_{\Theta_{-i}}
  (p_i(\theta)\theta_i - x_i(\theta))f_{-i}(\theta_{-i})\,d\theta_{-i}
\end{equation}
where $\theta$ is the vector of valuations of the $n$
bidders. Then, assuming the seller values the item at $0$, the expected revenue is
\begin{equation}
  \label{eq:3}
  U_0(p,x) = \int_\Theta \left(\sum_i x_i(\theta)\right)
  f(\theta)\, d\theta
\end{equation}
Clearly not every mechanism is feasible, so we impose the constraints
\begin{enumerate}
\item $\sum_i p_i(\theta) = 1$ and $p_i(\theta) \ge 0$
  for all $i$ and $\theta$, i.e. the allocation rule is a
  valid probability distribution.
\item (IR) $U_i(p,x,\theta_i) \ge 0$ for every agent and
  valuation $\theta_i$, i.e. participation leaves you at
  least as well off
\item (IC) $U_i(p,x,\theta_i) \ge \int_{\Theta_{-i}}
  (p_i(s_i,\theta_{-i}) \theta_i - x_i(s_i,\theta_{-i}))
  f_{-i}(\theta_{-i})\,d\theta_{-i}$, i.e. the agent does
  not want to misreport his valuation as $s_i$.
\end{enumerate}

Feasibility turns out to be equivalent to the following
conditions:
\begin{lemma}[Lemma 2 of Myerson 1981]
  A mechanism $(p,x)$ is feasible if and only if
  \begin{enumerate}
  \item The allocation probability from report $s_i$,
    denoted $Q_i(p,s_i) = \int_{\Theta_{-i}}
    p_i(s_i,\theta_{-i})
    f_{-i}(\theta_{-i})\,d\theta_{-i}$, is non-decreasing (ND)
      in the report.
    \item $U_i(p,x,\theta_i) = U_i(p,x,a_i) +
      \int_{a_i}^{\theta_i} Q_i(p,s_i) \, ds_i$ for every
      agent and every $s_i,\theta_i \in [a_i,b_i]$.
    \item $U_i(p,x,a_i) \ge 0$ for every agent.
    \item  $\sum_i p_i(\theta) = 1$ and $p_i(\theta) \ge 0$
  for all $i$ and $\theta$
  \end{enumerate}

\end{lemma}

Given this characterization, we have the following
theorem
\begin{theorem}
  An auction (p,x) that maximizes
  \begin{equation}
    \label{eq:4}
    \int_\Theta \left( \sum_i\left(\theta_i -
        \frac{1-F_i(\theta_i)}{f_i(\theta_i)}\right)p_i(\theta)\right) f(\theta)\,d\theta
  \end{equation}
  subject to the IC constraint and the non-decreasing
  constraint is the optimal auction if
  \begin{equation}
 x_i(t) = p_i(\theta)\theta_i - \int_{a_i}^{\theta_i} p_i(s_i,\theta_{-i})\,ds_i\label{eq:6}
 \end{equation}
\end{theorem}
In a typical application, you rewrite the objective
fucntion so it takes the form of the previous theorem and
maximize is subject to the IR constraint only. You'll
often end up wiht a solution that satisfies the ND
constraint iff $\theta_i -
\frac{1-F_i(\theta_i)}{f_i(\theta_i)}$ is non-decreasing.

%%%%%%%%%%%%%%%
\end{spacing}
\end{document}
%%%%% FIN %%%%%
%%%%%%%%%%%%%%%