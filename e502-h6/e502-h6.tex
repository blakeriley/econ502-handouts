\documentclass[letter, 10pt]{article}
% Change "article" to "report" to get rid of page number on title page
\usepackage{mathtools,amsfonts,amsthm,amssymb}
\usepackage{setspace}
\usepackage{graphicx,float,wrapfig}
%\usepackage{parskip}
\usepackage{enumerate}

% In case you need to adjust margins:
\topmargin=-0.45in      %
\evensidemargin=0in     %
\oddsidemargin=0in      %
\textwidth=6.5in        %
\textheight=9.0in       %
\headsep=0.25in         %
%\parindent=0in


%%%%%%%%%%%%%%%%%%%%%%%%%%%%%%%%%%%%%%%%%%%%%%%%%%%%%%%%%%%%%
% Format related commands


% Includes a figure
% The first parameter is the label, which is also the name of the figure
%   with or without the extension (e.g., .eps, .fig, .png, .gif, etc.)
%   IF NO EXTENSION IS GIVEN, LaTeX will look for the most appropriate one.
%   This means that if a DVI (or PS) is being produced, it will look for
%   an eps. If a PDF is being produced, it will look for nearly anything
%   else (gif, jpg, png, et cetera). Because of this, when I generate figures
%   I typically generate an eps and a png to allow me the most flexibility
%   when rendering my document.
% The second parameter is the width of the figure normalized to column width
%   (e.g. 0.5 for half a column, 0.75 for 75% of the column)
% The third parameter is the caption.
\newcommand{\scalefig}[3]{
  \begin{figure}[ht!]
    % Requires \usepackage{graphicx}
    \centering
    \includegraphics[width=#2\columnwidth]{#1}
    %%% I think \captionwidth (see above) can go away as long as
    %%% \centering is above
    %\captionwidth{#2\columnwidth}%
    \caption{#3}
    \label{#1}
  \end{figure}}


\newtheorem{theorem}{Theorem}[section]
\newtheorem{definition}[theorem]{Definition}
\newtheorem{lemma}[theorem]{Lemma}
\theoremstyle{definition}
\newtheorem{exercise}{Exercise}[section]

%%%%%%%%%%%%%%%%%%%%%%%%%%%%%%%%%%%%%%%%%%%%%%%%%%%%%%%%%%%%

% Custom commands

\newcommand{\twopartdef}[4]
{
	\left\{
		\begin{array}{ll}
			#1 & \mbox{if } #2 \\
			#3 & \mbox{if } #4
		\end{array}
	\right.
}

\newcommand{\R}{\mathbb{R}}
\DeclareMathOperator{\E}{E}
\newcommand{\tto}{\twoheadrightarrow}

\DeclareMathOperator{\SPA}{SPA}
\DeclareMathOperator{\FPA}{FPA}

%%%%%%%%%%%%%%%%%%%%%%%%%%%%%%%%%%%%%%%%%%%%%%%%%%%%%%%%%%%%%
%%%%%%%%%% The main document content
%%%%%%%%%%%%%%%%%%%%%%%%%%%%%%%%%%%%%%%%%%%%%%%%%%%%%%%%%%%%%

\begin{document}
\begin{spacing}{1.1}

\noindent
\textbf{Handout 6} \\
Econ 502 \\
March 8, 2013 \\
TA: Blake Riley \\

\section{Auction Theory}

Remember that if $v\in \R^n\,iid \sim F$, then
$P(v_i>v_j, \forall j \ne i) = F(v_i)^{n-1}$, so
the density of the largest order statistic of $n-1$ other
agents is $(n-1)F(v_{(1)})^{n-2} f(v_{(1)})$.


\subsection{Second-price Auction}

Denote the second-price auction with reserve price $r$
and entry fee $c$ as $\SPA(r,c)$. We have the following
result:

\begin{theorem}[$\SPA(r,0)$ in dominant strategies] The
  following consitutes a weakly dominant strategy for
  player $i$ in the $\SPA(r,0)$:
  \begin{equation}
    b(v) =
    \begin{cases}
       \mathrm{No} & \mbox{if } v < r \\
      v & \mbox{if } v \ge r
    \end{cases}
  \end{equation}
  where $ \mathrm{No}$ indicates no participation.
\end{theorem}

Let $v_0(r,c)$ be the $marginal value$, the valuation for
which a potential bidder is indifferent between
participating and not. This is part of the equilibrium,
and clearly depends on the distribution of values $F$.

\begin{theorem}[$\SPA(r,c)$]
  Assume $0 \le c \le 1-r \le 1$. Then, a BNE of
  the $\SPA(r,c)$ with $n$ bidders consists of each player using the
  following strategy:
    \begin{equation}
    b(v) =
    \begin{cases}
      \mathrm{No} & \mbox{if } v < v_0(r,c) \\
      v & \mbox{if } v \ge v_0(r,c)
    \end{cases}
  \end{equation}
  where $v_0(r,c)$ solves $(v_0 - r)F(v_0)^{n-1} = c$
\end{theorem}

\subsection{First-price Auction}

Denote the first-price auction with reserve price $r$ and
entry fee $c$ as $\FPA(r,c)$.

\begin{theorem}[Symmetric Equilibrium in $\FPA(0,0)$]
  Consider a $\FPA(0,0)$ with $n$ bidders and iid
  private values with distribution $F$. THen the
  following is a symmetric $BNE$ strategy:
  \begin{equation}
    \label{eq:1}
    b(v) = \int_0^v t \frac{g(t)}{G(v)}\,dt = v -
    \int_0^v \frac{G(t)}{G(v)}\, dt = \E[v_{(1)}\,|\,v_{(1)}<v]
  \end{equation}
  where $G(v) = F(v)^{n-1}$ and $v_{(1)}$ is the first
  order statistic among over $n-1$ agents. Moreover, $b$
  is strictly increasing on $[0,1]$ and is
  differentiable.
\end{theorem}

\section{Revenue Equivalence}

Myerson (1981) asks which mechanism maximizes the revenue of the seller
among all feasible mechanism (where feasible means implementable in
Bayesian Nash equilibrium and individually rational). It turns out that
just changing the format of the auction without changing who gets the item
won't make much of a difference:

\begin{theorem}[Revenue Equivalence]
  Suppose bidders are risk neutral and have iid private
  values. Then, all equilibria of any two auctions that generate
  the same allocation rule and the same conditional
  expected payoffs for each buyer with value $0$ produce
  the same expected revenue for the seller.
\end{theorem}
Revenue equivalence in auctions is a special case of the following theorem:
\begin{theorem}[General Revenue Equivalence]
  Suppose agents are risk neutral and have independent private types
  distributed on connected supports. Then, all equilibria of any two
  mechanisms that implement the same social choice function and give the
  same conditional expected payoffs for some type of each agent produce
  the same expected revenue.
\end{theorem}
Revenue equivalence has two important implications. First, we'll have to
depart from efficiency or IR if we want to increase revenue. Second, since
we don't observe empirical equivalence between first- and second-price
auctions, we'll need to invoke risk-aversion, common-values, or
non-expected-utility maximization to fully explain behavior in auctions.


%%%%%%%%%%%%%%%
\end{spacing}
\end{document}
%%%%% FIN %%%%%
%%%%%%%%%%%%%%%