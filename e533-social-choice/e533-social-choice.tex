\documentclass[10pt]{article}

\def\HandoutNumber{4}
\def\TheDate{February 18, 2016}
\def\Name{Blake Riley}

\usepackage{amsmath,amsfonts,amsthm,amssymb}
\usepackage{setspace}
\usepackage{graphicx,float,wrapfig}
%\usepackage{parskip}
\usepackage{enumerate}
\usepackage{url}

\usepackage{fourier}
\usepackage[T1]{fontenc}
\usepackage[protrusion=true,expansion=true]{microtype}

% In case you need to adjust margins:
\topmargin=-0.45in      %
\evensidemargin=0in     %
\oddsidemargin=0in      %
\textwidth=6.5in        %
\textheight=9.0in       %
\headsep=0.25in         %
\parindent=0in

\usepackage[nodayofweek]{datetime} \usdate
% Pdf metadata
\pdfinfo{  /Author (\Name)
           /Title (Economics 533 Handout \HandoutNumber)
           /Keywords ()
           /ModDate (D:\pdfdate)}

\newtheoremstyle{basic}% name
   {5pt}% Space above
   {5pt}% Space below
   {\itshape \leftskip=1em}% Body font
   {-1em}% Indent amount
   {\bfseries}% Theorem head font
   {:}% Punctuation after theorem head
   { }% Space after theorem head
   {}% Theorem head spec (can be left empty, meaning `normal')
\theoremstyle{basic}
\newtheorem{exercise}{Exercise}[]
\newtheorem{definition}{Definition}[section]
\newtheorem{theorem}{Theorem}[section]
\newtheorem{lemma}[theorem]{Lemma}

\usepackage{tikz} % For drawing diagrams
\usetikzlibrary{calc}
\tikzset{
% Two node styles for game trees: solid and hollow
solid node/.style={circle,draw,inner sep=1.5,fill=black},
hollow node/.style={circle,draw,inner sep=1.5}
}

%%%%%%%%%%%%%%%%%%%%%%%%%%%%%%%%%%%%%%%%%%%%%%%%%%%%%%%%%%%%

% Custom commands

\newcommand{\R}{\mathbb{R}}
\newcommand{\N}{\mathbb{N}}
\newcommand{\E}{\operatorname{E}}
\renewcommand{\P}{\operatorname{Pr}}
\newcommand{\Var}{\operatorname{Var}}
\newcommand{\Cov}{\operatorname{Cov}}
\newcommand{\cond}{\,|\,}
\newcommand{\bigcond}{\;\big|\;}
\newcommand{\argmax}{\mathop{\operatorname{arg\,max}}}
\newcommand{\noti}{{{\scriptscriptstyle-}\!i}}
\newcommand{\notj}{{{\scriptscriptstyle-}\!j}}
\newcommand{\notij}{{{\scriptscriptstyle-}\!\{i,j\}}}
\newcommand{\I}{\mathbb{I}}
\newcommand{\tto}{\twoheadrightarrow}


%%%%%%%%%%%%%%%%%%%%%%%%%%%%%%%%%%%%%%%%%%%%%%%%%%%%%%%%%%%%%
%%%%%%%%%% The main document content
%%%%%%%%%%%%%%%%%%%%%%%%%%%%%%%%%%%%%%%%%%%%%%%%%%%%%%%%%%%%%

\begin{document}
\begin{spacing}{1.0}

\noindent
\textbf{Handout \HandoutNumber} \\
Econ 533 \\
\TheDate \\
TA: \Name \\

\section{Social Choice Theory}

Social choice theory deals with the aggregation of
individual preferences over some number of
alternatives. The arising aggregated preferences then
reflect some sort of social preferences, and like
individual preferences, we want these to satisfy certain
conditions. For example, we may want unanimous agreement
on the best option to be maintained in the social ranking
by placing this option on top.

\subsection{Notation}

We consider the following $\mathcal{E} = \{X,
\succeq_i\}_{i=1}^n$ as representing a social group or
economy, where $n$ is the number of agents in the
economy. We consider identical sets of alternatives $X$
for each individual, and heterogeneous, rational
(complete and transitive) preference relations. We denote
with $R$ the set of all possible rational (weak)
preferences relations on $X$ and $P$ as the set of all
strict rational preferences. A typical element of $R
\times R \times \ldots \times R = R^n$ is $(\succeq_1,
\ldots, \succeq_n)$, which will be called a preference
profile. If $X$ is finite, then each profile is a
collection of $n$ rankings of the alternatives.

\section{Social Welfare Functions}

\begin{definition}
  A \textbf{social welfare function} is a rule $F: A\to
  R$ (with $A= R^n$ or $P^n$) that assigns a rational
  preferences relation $F(\succeq_1, \ldots, \succeq_n)
  \in R$, interpreted as the social preference relation,
  to any profile of individual rational preferences in $A$.
\end{definition}

\begin{definition}
  A social welfare function $F:A\to R$ is
  \textbf{Pareto-efficient} or
  \textbf{Paretian} if, for all alternatives
  $x,y \in X$ and all preference profiles
  $\succeq=(\succeq_1, \ldots, \succeq_n)\in A$, we have \[\forall
  i\in n, \, x \succ_i y \implies x \, F(\succeq) \, y\]
\end{definition}

\begin{definition}
  A social welfare function $F:A\to R$ satisfies
  \textbf{independence of irrelevant alternatives (IIA)}  if for
  all $x,y \in X$ and for all pairs of profiles
  $\succeq , \succeq'\, \in A$ with the property \[x\succeq_i y
  \iff x\succeq'_i y \quad\text{ and }\quad y \succeq_i x \iff y
  \succeq'_i x\] we have \[x\, F(\succeq) \, y \iff x
  \,F(\succeq')\, y \quad\text{ and }\quad y\, F(\succeq) \, x \iff y
  \,F(\succeq')\, x\]
\end{definition}

\begin{definition}
  A social welfare function $F: A\to R$ is
  \textbf{dictatorial} if there is an agent $h\in n$ such
  that for all $x,y \in X$ and for all $\succeq \,\in A$, we
  have $x \succeq_h \implies x\, F(\succeq) \, y$.
\end{definition}

\section{Social Choice Functions}

Social preferences aren't that useful unless they are
used to make some sort of social choice. Given that, we
can represent the process of making a choice from a set
of options based on a preference profile as a single function.

\begin{definition}
  A \textbf{social choice function} is a rule $f: A\to X$
  (with $A = R^I$ or $p^I$) that assigns a chosen element
  $f(\succeq_1, \ldots, \succeq_I) \in X$ to every profile
  of individual rational preference relations in $A$.
\end{definition}

At the moment, we are assuming a single-valued function,
but this can readily be extended to a correspondence following Maskin.

\begin{definition}
  A social choice function $f: A \to X$ is \textbf{weakly
    Pareto efficient} is, for all preference profiles
  $\succeq\, \in A$, the choice $f(\succeq) \in X$ is a
  weak Pareto optimum, i.e. for all $x,y\in X$ such that
  $x \succ_i y$ for all $i\in n$, then $y \not = f(\succeq)$.
\end{definition}

\begin{definition}
  The alternative $x\in X$ \textbf{maintains its
    position} from $\succeq\,\in R^n$ to $\succeq' \,\in
  R^n$ if $x \succeq_i y \implies x \succeq' y$ for all
  $i\in n$ and $y \in X$. Equivalently, the lower contour
  sets of $x$ in the preferences of the first profile are
  subsets of the lower contour sets of the corresponding
  preferences in the second profile.
\end{definition}

\begin{definition}
  A social choice function $f: A \to X$ is
  \textbf{monotonic} if for all profiles $\succeq,
  \succeq'\,\in A$ where $x = f(\succeq)$ maintains its
  position from $\succeq$ to $\succeq'$, we have
  $f(\succeq')=x$ again.
\end{definition}

\begin{definition}
  A social choice function is \textbf{dictatorial} if
  there exists an agent $h\in n$ such that for all
  profiles $\succeq \,\in A$, we have $f(\succeq) =
  \{x\;|\; \forall y\in X, x\succeq_h y\}$,
  i.e. $f(\succeq)$ is a maximal element for $h$ over $X$.
\end{definition}

\begin{definition}
  A social choice function is \textbf{strategy-proof} if
  for all $i\in n$, preferences profile $\succeq\,\in A$,
  and alternate preference $\succeq'_i \in R$, we have
  $f(\succeq_i, \succeq_{-i}) \succ_i f(\succeq'_i, \succeq_{-i})$.
\end{definition}

\section{Impossibility Results}

\begin{theorem}
  (Arrow 1950) Suppose $|X| \geq 3$ and $A = R^n$ or
  $P^n$. Then every social welfare function $F:A\to R$
  that is Pareto-efficient and satisfies IIA is dictatorial.
\end{theorem}

\begin{theorem}
  (Muller and Satterthwaite 1977) Suppose $|X| \geq 3$
  and $A = R^n$ or $P^n$. Then every social choice
  function $f:A \to X$ that is weakly Pareto-efficient
  and monotonic is dictatorial.
\end{theorem}

\begin{theorem}
  (Gibbard 1973, Satterthwaite 1977) If  $|X| \geq 3$ and
  the social choice function $f: A \to X$ is onto (surjective) and
  strategy-proof, then $f$ is dictatorial.
\end{theorem}

\newpage
\section{Exercises}
\label{sec:exercises}

\begin{exercise}[Priority Auction]

  Agents value an item at $\theta_i \sim \operatorname{Unif}[0,1]$ where values are
  private information. The item is going to be sold by the following rules: Agents
  have a simultaneous choice of purchasing a priority level $A$ or $B$ and submitting
  a bid $b_i \in [0,1]$. Choosing $A$ has a cost of $c$ while $B$ is free. A
  second-price auction is held within each priority level using the given bids,
  starting with $A$. Those with priority $B$ have a chance of winning only if no one
  chose priority $A$.

  \vspace{.5em}
  Construct an equivalent revelation mechanism for two agents.
  
\end{exercise}

\begin{exercise}[Sprumont 1991]
  One useful restriction of preferences is \emph{single-peakedness}. If the
  set of options $X$ is single-dimensional and ordered, then a preference
  ordering $\succ$ on $X$ is single-peaked iff there is a maximum $x^*$ of
  $\succ$ and $x^* < x < y$ or $y < x < x^* \implies x \succ y$.

  \vspace{.5em}
  Consider a partnership of $n$ individuals who will invest in a project,
  with the benefits shared in proportion to each partner's investment. The
  project has a fixed cost of $1$. The partners have single-peaked
  preferences over the amount they want to invest, with a peaks $x_i^* \in
  [0,1]$. Because the sum of the peak amount of all partners may not be
  equal to the cost of the project, some partners may be forced to invest
  more or less than their ideal amounts. In this context, efficiency means
  that if the sum of ideal investments is less than the cost, everyone must
  invest at least their ideal amount, and vice versa.

  \vspace{.5em}
  In addition to strategy-proofness and efficiency, let's consider two
  other desirable properties of social choice functions. First, an scf is
  \emph{anonymous} if $f(\theta)=f(\pi(\theta))$ for all permutations $\pi$
  of the vector. Second, an scf is \emph{envy-free} if for all $i,j$,
  $f_i(\theta) \succeq_i f_j(\theta)$, i.e. the proportion allocated to $i$
  is preferred by $i$ to all other agents' allocations.

  \vspace{.5em}
  Check whether the following are strategy-proof,
  efficient, anonymous, and/or envy-free:
  \begin{enumerate}
  \item The egalitarian rule $f_i^e(\theta) = 1/n$.
  \item The proportional rule $f_i^p(\theta) = x_i^*/\sum x_j^*$.
  \item The priority rule $f_1^q(\theta) = x_1^*$ and
    $f_i^q(\theta) = \min\{x_i^*, 1-\sum_{j<i} x_j^*\}$
    when $\sum x_i^* \geq 1$, and alternately $f_j^q(\theta) = x_j^*$
    for $j < n$ and $f_n^q(\theta) = 1 - \sum_{j=1}^{n-1}
    x_j^*$ when $\sum x^i < 1$.
  \item The priority rule as defined above, but with a random order.
  \item The uniform rule defined by the iterative process (for the case of
    $\sum x_i^* > 1$):
    \begin{enumerate}
    \item Start with all partners active and the full cost outstanding.
    \item Divide the outstanding cost equally among the active partners.
    \item If any active partner has an ideal below the equal
      share, set their share equal to their ideal and
      subtract this from the outstanding cost. These
      partners are now inactive.
    \item Repeat the previous two steps among the active
      partners until each has an ideal amount no less
      than the equal share of the outstanding cost.
    \end{enumerate}

  \end{enumerate}

\end{exercise}


%%%%%%%%%%%%%%%
\end{spacing}
\end{document}
%%%%% FIN %%%%%
%%%%%%%%%%%%%%%