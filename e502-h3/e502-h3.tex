\documentclass[letter, 10pt]{article}
% Change "article" to "report" to get rid of page number on title page
\usepackage{amsmath,amsfonts,amsthm,amssymb}
\usepackage{setspace}
\usepackage{graphicx,float,wrapfig}
%\usepackage{parskip}
\usepackage{enumerate}

% In case you need to adjust margins:
\topmargin=-0.45in      %
\evensidemargin=0in     %
\oddsidemargin=0in      %
\textwidth=6.5in        %
\textheight=9.0in       %
\headsep=0.25in         %
%\parindent=0in


%%%%%%%%%%%%%%%%%%%%%%%%%%%%%%%%%%%%%%%%%%%%%%%%%%%%%%%%%%%%%
% Format related commands


% Includes a figure
% The first parameter is the label, which is also the name of the figure
%   with or without the extension (e.g., .eps, .fig, .png, .gif, etc.)
%   IF NO EXTENSION IS GIVEN, LaTeX will look for the most appropriate one.
%   This means that if a DVI (or PS) is being produced, it will look for
%   an eps. If a PDF is being produced, it will look for nearly anything
%   else (gif, jpg, png, et cetera). Because of this, when I generate figures
%   I typically generate an eps and a png to allow me the most flexibility
%   when rendering my document.
% The second parameter is the width of the figure normalized to column width
%   (e.g. 0.5 for half a column, 0.75 for 75% of the column)
% The third parameter is the caption.
\newcommand{\scalefig}[3]{
  \begin{figure}[ht!]
    % Requires \usepackage{graphicx}
    \centering
    \includegraphics[width=#2\columnwidth]{#1}
    %%% I think \captionwidth (see above) can go away as long as
    %%% \centering is above
    %\captionwidth{#2\columnwidth}%
    \caption{#3}
    \label{#1}
  \end{figure}}

\newtheorem{exercise}{Exercise}[section]
\newtheorem{definition}{Definition}[section]
\newtheorem{theorem}{Theorem}[section]
\newtheorem{lemma}[theorem]{Lemma}

%%%%%%%%%%%%%%%%%%%%%%%%%%%%%%%%%%%%%%%%%%%%%%%%%%%%%%%%%%%%

% Custom commands

\newcommand{\R}{\mathbb{R}}
\newcommand{\E}{\operatorname{E}}

%%%%%%%%%%%%%%%%%%%%%%%%%%%%%%%%%%%%%%%%%%%%%%%%%%%%%%%%%%%%%
%%%%%%%%%% The main document content
%%%%%%%%%%%%%%%%%%%%%%%%%%%%%%%%%%%%%%%%%%%%%%%%%%%%%%%%%%%%%

\begin{document}
\begin{spacing}{1.1}

\noindent
\textbf{Handout 3} \\
Econ 502 \\
February 7, 2012 \\
TA: Blake Riley \\

\section{Finitely repeated games}

Unsurprisingly, in repeated games we're analyzing a
specific game (such as the prisoners' dilemma, battle of
the sexes, etc) that is played over and over again in a
supergame. At each stage of the repetition, let $G$
denote a static game of complete information in which the
players involved choose their corresponding actions
simultaneously. The game $G$ will then be called the
\textbf{stage game} of the full game.

\begin{definition}
  Given a stage game $G$, let $G(T)$ denote the finitely
  repeated game in which $G$ is played $T<\infty$ times,
  with the outcomes of all preceding plays observed
  before the next play begins. The payoffs of $G(T)$ are
  simply the sum of the payoffs for each of the $T$ stages.
\end{definition}

\begin{theorem}
  If the stage game $G$ has a unique Nash equilibrium,
  then for all finite $T$, the repeated game $G(T)$ has a
  unique subgame-perfect outcome where the unique NE is
  played at every stage.
\end{theorem}

We have seen that if there are multiple equilibria at every
stage, then there might be subgame-perfect NEs that do
not involve playing ``stage-game equilibrium
strategies''. Remember the example from last class where
this occurred with the two-stage game.

\section{Infinitely repeated games}

In finitely repeated games, there might be credible threats that influence
current behavior if the stage game has multiple equilibria. A stronger
result can be stated for the infinitely repeated case: even if the stage
game has a unique NE, there might be subgame-perfect NEs of the infinitely
repeated game in which no choice of actions in any stage is a NE of $G$.

\begin{definition}
  Given a stage game $G$, let $G(\infty, \delta)$ denote
  the infinitely repeated game in which $G$ is repeated
  indefinitely and players share the discount factor
  $\delta$. For each $t$, the outcomes the the $t-1$
  preceding plays are observed before the stage
  begins. The payoffs of each player are the present
  values of the sequence of payoffs from the stage games.
\end{definition}

\subsection{Payoffs}
 In contrast to simply summing up payoffs like in the
 finitely repeated games, we're now discounting future
 payoffs to make the near future more salient and to keep
 things tractable. Given a discount factor $\delta$ and a
 sequence of payoffs $\{\pi_t^i\}_{t=1}^\infty$ to player
 $i$, the total payoff is
 $\pi_1^i+\delta\pi_2^i+\delta^2\pi_3^i+\ldots = \sum
 \delta^{t-1} \pi_t$. Occasionally this is rescaled by
 $1-\delta$ so we get rid of this factor when stage game
 payoffs are identical across all periods.

\subsection{Folk theorems}

\begin{definition}
  In an $n$-player game, call the vector $(x_1,\ldots,
  x_n)$ a \textit{feasible vector} in stage game $G$ if each $x_i$
  is a convex combination of the pure-strategy payoffs of $G$.
\end{definition}

\begin{theorem}
  (Friedman 1971) Let $G$ be a finite, static game of
  complete information with $n$ players. Let $(e_1,
  \ldots, e_n)$ be the payoffs from a Nash equilibrium of
  $G$, and let $(x_1, \ldots, x_n)$ be any other feasible
  payoffs from $G$, If $x_1 > e_i$ for each player $i$
  and $\delta$ is sufficiently close to one, then there
  exists a subgame-perfect Nash equilibrium of the
  infinitely repeated game $G(\infty, \delta)$ that
  achieves $(x_1, \ldots, x_n)$ as payoffs.
\end{theorem}


%%%%%%%%%%%%%%%
\end{spacing}
\end{document}
%%%%% FIN %%%%%
%%%%%%%%%%%%%%%