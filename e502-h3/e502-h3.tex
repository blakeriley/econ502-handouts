\documentclass[letter, 10pt]{article}
% Change "article" to "report" to get rid of page number on title page
\usepackage{amsmath,amsfonts,amsthm,amssymb}
\usepackage{setspace}
\usepackage{graphicx,float,wrapfig}
%\usepackage{parskip}
\usepackage{enumerate}

\usepackage{fourier}
\usepackage[T1]{fontenc}
\usepackage[protrusion=true,expansion=true]{microtype}

% In case you need to adjust margins:
\topmargin=-0.45in      %
\evensidemargin=0in     %
\oddsidemargin=0in      %
\textwidth=6.5in        %
\textheight=9.0in       %
\headsep=0.25in         %
\parindent=0in

\usepackage[nodayofweek]{datetime} \usdate
% Pdf metadata
\pdfinfo{  /Author (Blake Riley)
           /Title (Econ 502 Handout 3)
           /Keywords ()
           /ModDate (D:\pdfdate)}

\newtheoremstyle{basic}% name
   {5pt}% Space above
   {5pt}% Space below
   {\itshape \leftskip=1em}% Body font
   {-1em}% Indent amount
   {\bfseries}% Theorem head font
   {:}% Punctuation after theorem head
   { }% Space after theorem head
   {}% Theorem head spec (can be left empty, meaning `normal')
\theoremstyle{basic}
\newtheorem{exercise}{Exercise}[section]
\newtheorem{definition}{Definition}[section]
\newtheorem{theorem}{Theorem}[section]
\newtheorem{lemma}[theorem]{Lemma}


%%%%%%%%%%%%%%%%%%%%%%%%%%%%%%%%%%%%%%%%%%%%%%%%%%%%%%%%%%%%

% Custom commands

\newcommand{\R}{\mathbb{R}}
\newcommand{\N}{\mathbb{N}}
\newcommand{\E}{\operatorname{E}}
\renewcommand{\P}{\operatorname{Pr}}
\newcommand{\Var}{\operatorname{Var}}
\newcommand{\Cov}{\operatorname{Cov}}
\newcommand{\cond}{\,|\,}
\newcommand{\bigcond}{\;\big|\;}
\newcommand{\argmax}{\mathop{\operatorname{arg\,max}}}
\newcommand{\noti}{{{\scriptscriptstyle-}\!i}}
\newcommand{\notj}{{{\scriptscriptstyle-}\!j}}
\newcommand{\notij}{{{\scriptscriptstyle-}\!\{i,j\}}}
\newcommand{\I}{\mathbb{I}}

%%%%%%%%%%%%%%%%%%%%%%%%%%%%%%%%%%%%%%%%%%%%%%%%%%%%%%%%%%%%%
%%%%%%%%%% The main document content
%%%%%%%%%%%%%%%%%%%%%%%%%%%%%%%%%%%%%%%%%%%%%%%%%%%%%%%%%%%%%

\begin{document}
\begin{spacing}{1.0}

\noindent
\textbf{Handout 3} \\
Econ 502 \\
February 14, 2014 \\
TA: Blake Riley \\

\section{Finitely repeated games}

Unsurprisingly, in repeated games we're analyzing a
specific game (such as the prisoners' dilemma, battle of
the sexes, etc) that is played over and over again in a
supergame. At each stage of the repetition, let $G$
denote a static game of complete information in which the
players involved choose their corresponding actions
simultaneously. The game $G$ will then be called the
\textbf{stage game} of the full game.

\begin{definition}
  Given a stage game $G$, let $G(T)$ denote the finitely
  repeated game in which $G$ is played $T<\infty$ times,
  with the outcomes of all preceding plays observed
  before the next play begins. The payoffs of $G(T)$ are
  simply the sum of the payoffs for each of the $T$ stages.
\end{definition}

\begin{theorem}
  If the stage game $G$ has a unique Nash equilibrium,
  then for all finite $T$, the repeated game $G(T)$ has a
  unique subgame-perfect outcome where the unique NE is
  played at every stage.
\end{theorem}

If there are multiple equilibria at every
stage, then there might be subgame-perfect NEs that do
not involve playing ``stage-game equilibrium
strategies''.

\section{Infinitely repeated games}

In finitely repeated games, there might be credible threats that influence
current behavior if the stage game has multiple equilibria. A stronger
result can be stated for the infinitely repeated case: even if the stage
game has a unique NE, there might be subgame-perfect NEs of the infinitely
repeated game in which no choice of actions at any stage is a NE of $G$.

\begin{definition}
  Given a stage game $G$, let $G(\infty, \delta)$ denote
  the infinitely repeated game in which $G$ is repeated
  indefinitely and players share the discount factor
  $\delta$. For each $t$, the outcomes the the $t-1$
  preceding plays are observed before the stage
  begins. The payoffs of each player are the present
  values of the sequence of payoffs from the stage games.
\end{definition}

\subsection{Payoffs}
 In contrast to simply summing up payoffs like in the
 finitely repeated games, we're now discounting future
 payoffs to make the near future more salient and to keep
 things tractable. Given a discount factor $\delta$ and a
 sequence of payoffs $\{\pi_t^i\}_{t=1}^\infty$ to player
 $i$, the total payoff is
 $\pi_1^i+\delta\pi_2^i+\delta^2\pi_3^i+\ldots = \sum
 \delta^{t-1} \pi_t$. Occasionally this is rescaled by
 $1-\delta$ so we get rid of this factor when stage game
 payoffs are identical across all periods.

 The discount factor $\delta$ can often be interpreted as either a
time-discount factor or a fixed probability that the game ends at that stage.


\subsection{Folk theorems}

A ``folk theorem'' is a name for any theorem that is generally known, but
not necessesarily attributable to any individual. In the context of
repeated games, ``the folk theorem'' is a general feasibility theorem that
says a very large range of payoffs are acheivable in a Nash equilibrium of
an infinitely repeated game. Many further extensions exist for folk
theorems when agents have incomplete information and actions are only
partially observable.

\begin{definition}
  In an $n$-player game, call the vector $(x_1,\ldots,
  x_n)$ a \emph{feasible vector} in stage game $G$ if each $x_i$
  is a convex combination of the pure-strategy payoffs of $G$.
\end{definition}

\begin{definition}
  A payoff vector $(x_1, \ldots, x_n)$ is \emph{enforceable} if, for all $i$, the
  payoff $x_i$ is at least the minimum payoff other players can force on $i$:
  \begin{align*}
    x_i \ge m_i = \min_{s_\noti} \max_{s_i} \pi_i(s_i, s_\noti)
  \end{align*}
\end{definition}

\begin{theorem}
  (``The Folk Theorem'') Let $G$ be a finite, static game of complete
  information with $n$ players, and let $(x_1, \ldots, x_n)$ be a feasible
  and enforceable payoff vector. Then there exists a Nash equilibrium of
  $G(\infty, \delta)$ that acheives $x$ as payoffs if $\delta$ is
  sufficiently close to one.
\end{theorem}

\begin{theorem}
  (Friedman 1971) Let $G$ be a finite, static game of
  complete information with $n$ players. Let $(e_1,
  \ldots, e_n)$ be the payoffs from a Nash equilibrium of
  $G$, and let $(x_1, \ldots, x_n)$ be any other feasible
  payoffs from $G$. If $x_1 > e_i$ for each player $i$
  and $\delta$ is sufficiently close to one, then there
  exists a subgame-perfect Nash equilibrium of the
  infinitely repeated game $G(\infty, \delta)$ that
  achieves $(x_1, \ldots, x_n)$ as payoffs.
\end{theorem}

\begin{theorem}
  (Aumann and Shapley 1976, Rubinstein 1979) Even without discounting,
  every feasible and enforceable vector of payoffs is acheivable in a
  subgame perfect Nash equilibrium.
\end{theorem}



%%%%%%%%%%%%%%%
\end{spacing}
\end{document}
%%%%% FIN %%%%%
%%%%%%%%%%%%%%%