\documentclass[letter, 10pt]{article}
% Change "article" to "report" to get rid of page number on title page
\usepackage{amsmath,amsfonts,amsthm,amssymb}
\usepackage{setspace}
\usepackage{graphicx,float,wrapfig}
%\usepackage{parskip}
\usepackage{enumerate}

% In case you need to adjust margins:
\topmargin=-0.45in      %
\evensidemargin=0in     %
\oddsidemargin=0in      %
\textwidth=6.5in        %
\textheight=9.0in       %
\headsep=0.25in         %
%\parindent=0in


%%%%%%%%%%%%%%%%%%%%%%%%%%%%%%%%%%%%%%%%%%%%%%%%%%%%%%%%%%%%%
% Format related commands


% Includes a figure
% The first parameter is the label, which is also the name of the figure
%   with or without the extension (e.g., .eps, .fig, .png, .gif, etc.)
%   IF NO EXTENSION IS GIVEN, LaTeX will look for the most appropriate one.
%   This means that if a DVI (or PS) is being produced, it will look for
%   an eps. If a PDF is being produced, it will look for nearly anything
%   else (gif, jpg, png, et cetera). Because of this, when I generate figures
%   I typically generate an eps and a png to allow me the most flexibility
%   when rendering my document.
% The second parameter is the width of the figure normalized to column width
%   (e.g. 0.5 for half a column, 0.75 for 75% of the column)
% The third parameter is the caption.
\newcommand{\scalefig}[3]{
  \begin{figure}[ht!]
    % Requires \usepackage{graphicx}
    \centering
    \includegraphics[width=#2\columnwidth]{#1}
    %%% I think \captionwidth (see above) can go away as long as
    %%% \centering is above
    %\captionwidth{#2\columnwidth}%
    \caption{#3}
    \label{#1}
  \end{figure}}

\newtheorem{exercise}{Exercise}[section]
\newtheorem{definition}{Definition}[section]
\newtheorem{theorem}{Theorem}[section]
\newtheorem{lemma}[theorem]{Lemma}

%%%%%%%%%%%%%%%%%%%%%%%%%%%%%%%%%%%%%%%%%%%%%%%%%%%%%%%%%%%%

% Custom commands

\newcommand{\twopartdef}[4]
{
	\left\{
		\begin{array}{ll}
			#1 & \mbox{if } #2 \\
			#3 & \mbox{if } #4
		\end{array}
	\right.
}

\newcommand{\R}{\mathbb{R}}
\newcommand{\E}{\operatorname{E}}

%%%%%%%%%%%%%%%%%%%%%%%%%%%%%%%%%%%%%%%%%%%%%%%%%%%%%%%%%%%%%
%%%%%%%%%% The main document content
%%%%%%%%%%%%%%%%%%%%%%%%%%%%%%%%%%%%%%%%%%%%%%%%%%%%%%%%%%%%%

\begin{document}
\begin{spacing}{1.1}

\noindent
\textbf{Handout 2} \\
Econ 502 \\
February 1, 2012 \\
TA: Blake Riley \\

\begin{center}
{\Large Microeconomic Theory I: Spring 2013}
\end{center}

\section{Refinements and Dynamic Games}

\subsection{Trembling-hand perfection}

\begin{definition}
  A Nash equilibrium $\sigma$ of a normal form game is
  \textbf{trembling-hand perfect} iff there is some sequence of totally
  mixed strategies $\{\sigma^k\}_{k=1}^\infty$ such that $\forall i,k$,
  $\lim \sigma^k = \sigma$ and $\sigma_i$ is a best response to every
  element of $\sigma_{-i}^k$.
\end{definition}

\begin{lemma}
  If $\sigma$ is a normal form trembling- hand perfect Nash equlibrium, then
  $\sigma_i$ is not a weakly dominated strategy for any $i$. Hence, no weakly
  dominated pure strategy is played with positive probability.
\end{lemma}

\begin{theorem}
  Every game with finite strategy sets in normal form has a trembling-hand
  perfect Nash equilibrium.
\end{theorem}

\subsection{Subgame perfect Nash equilibria}

\begin{definition}
  A \textbf{subgame} of an extensive form game is a subset of the
  game having the following properties:
  \begin{itemize}
  \item It begins with an information set containing a
    single node (since games start from a single node).
  \item It contains all the nodes that are successors of
    this node and only these nodes.
  \item If a node is in the subgame, then all nodes in
    the same information set are in the subgame (we can't
    cut through information sets).
  \end{itemize}
\end{definition}

\begin{definition}
  A strategy profile $\sigma = (\sigma_1, \ldots, \sigma_N)$ in an
  extensive form game is a \textbf{subgame perfect Nash equilibrium (SPNE)}
  of this game if it induces a Nash equilibrium in every subgame of the
  game.
\end{definition}

Every SPNE is a NE, but not vice versa (which is why it's known as a
refinement). If the only subgame of an extensive form game is the game
itself, then every NE is a SPNE.

\subsection{Identifying SPNEs}

Backward induction:

\begin{enumerate}
\item Identify how many subgames you have. If the only subgame is the game
  itself, every NE is an SPNE. In simple cases, drawing the normal form and
  finding NE is the best route. If this is applicable, stop here.
\item Go to the final subgames and find the NE of the final subgames.
\item Pick one of the equilibria and replace the final subgames by the
  corresponding equilibrium payoff vectors.
\item Now start the recursion. Do the previous steps again for the reduced
  game. If you find multiple equilibria at any step, each of the possible
  combinations constitutes a SPNE.
\item If there is a unique equilibrium strategy for each player at each
  possible subgame, the SPNE is unique.
\end{enumerate}

\begin{lemma}
  In a finite extensive form game, if no player has the same payoffs at any
  two terminal nodes, there is a unique Nash equilibrium that can be
  derived by backward induction. This is then also the unique SPNE.
\end{lemma}

\subsection{Weak perfect Bayesian Nash equilibria}

\begin{definition}
  A profile of strategies and systems of beliefs $(\sigma, \mu)$ is a
  \textbf{weak perfect Bayesian Nash equilibrium (WPBE)} in an extensive
  form game if it has the following properties:
  \begin{itemize}
  \item The strategy profile $\sigma$ is sequentially rational given given
    belief system $\mu$, i.e. once an information set is reached on the
    equilibrium path, no player finds it worthwhile to revise his strategy.
  \item The systems of beliefs $\mu$ is derived from strategy profile
    $\sigma$ using Bayes rule whenever possible.
  \end{itemize}
\end{definition}

The probability of being at one specific node in an information set that
lies on the equilibrium path has to be derived by Bayes rule. Notice
however, we don't put any restrictions on how to derive beliefs for
information sets not on the equilibrium path.

WPBE are NE, but not vice versa.

\subsection{How to identify WPBEs}

\begin{enumerate}
\item Check whether each player has some dominant strategies. This can save
  a significant amount of time.
\item Assign general beliefs to each node with non-singleton information
  sets and apply ``sequential rationality,'' i.e. compute expected payoffs
  of each possible action at this information set using the general
  beliefs. Find out which action has the higher expected payoff for which
  threshold beliefs. You will probably have several cases.
\item Apply sequential rationality to each singleton information set given
  the cases of the other information sets.
\item Update the beliefs that are on the equilibrium paths and ensure
  consistency with the assumed beliefs from step 2.
\end{enumerate}

\subsection{Sequential equilibria}

\begin{definition}
  A profile of strategies and systems of belief $(\sigma, \mu)$ is a
  \textbf{sequential equilibrium} (denoted elsewhere, perhaps more
  consistently, as \textbf{perfect Bayesian equilibria}) in an extensive
  form game if it has the following properties:
  \begin{itemize}
  \item The strategy profile $\sigma$ is sequentially rational given belief
    system $\mu$.
  \item There exists a sequence of completely mixed strategies
    $\left\{\sigma^k\right\}_{k=1}^\infty$ with $\lim_{k \to \infty} \sigma^k =
    \sigma$ such that $\mu = \lim_{k \to \infty} \mu^k$, where $\mu^k$ denotes
    the beliefs derives from strategy profile $\sigma^k$ using Bayes rule.
  \end{itemize}
\end{definition}

Sequential equilibria are WPBE and SPNE.

\subsection{Identifying sequential equilibria}

\begin{enumerate}
\item First you have to identify all WPBEs. If there is a unique WPBE and
  all information sets of all players are on the equilibrium path, then it
  is also a sequential equilibrium.
\item If not all information sets are reached: try to construct a sequence
  of mixed strategies that converge to the equilibrium strategies so that
  the induced beliefs converge to the beliefs given by the WPBE. If you
  find one such sequence you are done.
\item To show that something is not a sequential equilibrium can be
  difficult using the sequence approach. You would have to prove that for
  every sequence of mixed strategies, the induced beliefs don't converge to
  the proposed beliefs. A better approach: try to show that it's not a
  SPNE. If this is the case, then it can't be a sequential equilibrium.
\end{enumerate}

\subsection{Tips}

\begin{itemize}
\item Every sequential is a WPBE and a SPNE (so which sets of equilibria
  are larger)?
\item Not every WPBE is a SPNE.
\item If you find a sequence that does not converge to a pair $(\sigma,
  \mu)$, then you have not shown this is not a sequential equilibrium!
\item If for a pair $(\sigma, \mu)$ every information set is reached and
  $(\sigma, \mu)$ is a WPBE, then it is also a sequential equilibrium (and
  hence also a SPNE).
\item If you show that $(\sigma, \mu)$ is a WPBE, but not a SPBE, then it
  is not a sequential equilibrium.
\end{itemize}

%%%%%%%%%%%%%%%
\end{spacing}
\end{document}
%%%%% FIN %%%%%
%%%%%%%%%%%%%%%